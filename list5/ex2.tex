\documentclass{standalone}

\usepackage[polish]{babel}
\usepackage[utf8]{inputenc}
\usepackage[T1]{fontenc}
\usepackage{tikz}
\usetikzlibrary{er, positioning, arrows.meta}

\begin{document}
\begin{tikzpicture}[sibling distance=2cm, auto]
    \node[entity](sala){sala}
    [grow=up]
    child {node[attribute]{numer}}
    child {node[attribute]{budynek}};
    \node[relationship](ma miejsce)[below right = of sala]{ma miejsce};
    \node[entity](grupa)[below right = of ma miejsce]{grupa}
    [grow=down]
    child {node[attribute]{numer}}
    child {node[attribute]{dzień}}
    child {node[attribute]{godzina}}
    child {node[attribute]{czas}};
    \node[relationship](składa się)[above right = of grupa]{składa się};
    \node[relationship](prowadzi)[right = of grupa]{prowadzi};
    \node[entity](kurs)[above right = of składa się]{kurs}
    [grow=up]
    child {node[attribute]{nazwa}}
    child {node[attribute]{rodzaj}};
    \node[entity](prowadzący)[right = of prowadzi]{prowadzący}
    [grow=right]
    child {node[attribute]{imię}}
    child {node[attribute]{nazwisko}};
    \node[relationship](należy)[left = of grupa]{należy};
    \node[entity](student)[left = of należy]{student}
    [grow=left, level distance=3cm]
    child {node[attribute]{imię}}
    child {node[attribute]{nazwisko}}
    child {node[attribute]{indeks}};
    \path(składa się) edge node {} (grupa) edge[double distance=1.5pt] node {} (kurs);
    \path(prowadzi) edge node {} (grupa) edge[-Latex] node {} (prowadzący);
    \path(należy) edge[-Latex] node {} (grupa) edge node {} (student);
    \path(ma miejsce) edge[-Latex] node {} (sala) edge node {} (grupa);
\end{tikzpicture}
\end{document}