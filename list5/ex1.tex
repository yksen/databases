\documentclass[border=10pt]{standalone}

\usepackage[polish]{babel}
\usepackage[utf8]{inputenc}
\usepackage[T1]{fontenc}
\usepackage{tikz}
\usetikzlibrary{er, positioning, arrows.meta}

\begin{document}
\begin{tikzpicture}[auto]
    \node[entity](zamówienie){zamówienie}
    [grow=up]
    child {node[attribute]{data}};
    \node[relationship](złożył)[above left = of zamówienie]{złożył};
    \node[entity](klient)[above left = of złożył]{klient}
    [grow=up, sibling distance=2cm]
    child {node[attribute]{telefon}}
    child {node[attribute]{nazwa}}
    child {node[attribute]{adres}
            [grow=up, sibling distance=2cm]
            child {node[attribute]{ulica}}
            child {node[attribute]{miasto}}
            child {node[attribute]{kod}}};
    \path(złożył) edge[double distance=1.5pt] node {} (zamówienie);
    \draw[-Latex] (złożył) -- (klient);
    \node[relationship](obejmuje)[above right = of zamówienie]{obejmuje}
    [below right = of obejmuje]
    child {node[attribute]{sztuk}};
    \node[entity](produkt)[above right = of obejmuje]{produkt}
    [grow=up, sibling distance=2cm]
    child {node[attribute]{nazwa}}
    child {node[attribute]{ilość}}
    child {node[attribute]{cena}}
    child {node[attribute]{kategoria}};
    \path(obejmuje) edge node {} (zamówienie) edge node {} (produkt);
    \node[relationship](posiada)[below left = of klient]{posiada};
    \node[entity](zwierzę)[below left = of posiada]{zwierzę}
    [grow=down, sibling distance=2.2cm]
    child {node[attribute]{imię}}
    child {node[attribute]{gatunek}}
    child {node[attribute, dashed]{data ur.}}
    child {node[attribute, dashed]{wiek}};
    \path(posiada) edge node {} (zwierzę);
    \draw[-Latex] (posiada) -- (klient);
\end{tikzpicture}
\end{document}